\documentclass[]{book}
\usepackage{lmodern}
\usepackage{amssymb,amsmath}
\usepackage{ifxetex,ifluatex}
\usepackage{fixltx2e} % provides \textsubscript
\ifnum 0\ifxetex 1\fi\ifluatex 1\fi=0 % if pdftex
  \usepackage[T1]{fontenc}
  \usepackage[utf8]{inputenc}
\else % if luatex or xelatex
  \ifxetex
    \usepackage{mathspec}
  \else
    \usepackage{fontspec}
  \fi
  \defaultfontfeatures{Ligatures=TeX,Scale=MatchLowercase}
\fi
% use upquote if available, for straight quotes in verbatim environments
\IfFileExists{upquote.sty}{\usepackage{upquote}}{}
% use microtype if available
\IfFileExists{microtype.sty}{%
\usepackage{microtype}
\UseMicrotypeSet[protrusion]{basicmath} % disable protrusion for tt fonts
}{}
\usepackage[margin=1in]{geometry}
\usepackage{hyperref}
\hypersetup{unicode=true,
            pdftitle={GerminaQuant App},
            pdfauthor={Flavio Lozano Isla; Omar Benites Alfaro; Marcelo Francisco Pompelli},
            pdfborder={0 0 0},
            breaklinks=true}
\urlstyle{same}  % don't use monospace font for urls
\usepackage{natbib}
\bibliographystyle{apalike}
\usepackage{longtable,booktabs}
\usepackage{graphicx,grffile}
\makeatletter
\def\maxwidth{\ifdim\Gin@nat@width>\linewidth\linewidth\else\Gin@nat@width\fi}
\def\maxheight{\ifdim\Gin@nat@height>\textheight\textheight\else\Gin@nat@height\fi}
\makeatother
% Scale images if necessary, so that they will not overflow the page
% margins by default, and it is still possible to overwrite the defaults
% using explicit options in \includegraphics[width, height, ...]{}
\setkeys{Gin}{width=\maxwidth,height=\maxheight,keepaspectratio}
\IfFileExists{parskip.sty}{%
\usepackage{parskip}
}{% else
\setlength{\parindent}{0pt}
\setlength{\parskip}{6pt plus 2pt minus 1pt}
}
\setlength{\emergencystretch}{3em}  % prevent overfull lines
\providecommand{\tightlist}{%
  \setlength{\itemsep}{0pt}\setlength{\parskip}{0pt}}
\setcounter{secnumdepth}{5}
% Redefines (sub)paragraphs to behave more like sections
\ifx\paragraph\undefined\else
\let\oldparagraph\paragraph
\renewcommand{\paragraph}[1]{\oldparagraph{#1}\mbox{}}
\fi
\ifx\subparagraph\undefined\else
\let\oldsubparagraph\subparagraph
\renewcommand{\subparagraph}[1]{\oldsubparagraph{#1}\mbox{}}
\fi
\usepackage{booktabs}
\usepackage{longtable}
\usepackage{framed,color}
\definecolor{shadecolor}{RGB}{248,248,248}

\ifxetex
  \usepackage{letltxmacro}
  \setlength{\XeTeXLinkMargin}{1pt}
  \LetLtxMacro\SavedIncludeGraphics\includegraphics
  \def\includegraphics#1#{% #1 catches optional stuff (star/opt. arg.)
    \IncludeGraphicsAux{#1}%
  }%
  \newcommand*{\IncludeGraphicsAux}[2]{%
    \XeTeXLinkBox{%
      \SavedIncludeGraphics#1{#2}%
    }%
  }%
\fi

\newenvironment{rmdblock}[1]
  {\begin{shaded*}
  \begin{itemize}
  \renewcommand{\labelitemi}{
    \raisebox{-.7\height}[0pt][0pt]{
      {\setkeys{Gin}{width=3em,keepaspectratio}\includegraphics{images/#1}}
    }
  }
  \item
  }
  {
  \end{itemize}
  \end{shaded*}
  }
\newenvironment{rmdnote}
  {\begin{rmdblock}{note}}
  {\end{rmdblock}}
\newenvironment{rmdcaution}
  {\begin{rmdblock}{caution}}
  {\end{rmdblock}}
\newenvironment{rmdimportant}
  {\begin{rmdblock}{important}}
  {\end{rmdblock}}
\newenvironment{rmdtip}
  {\begin{rmdblock}{tip}}
  {\end{rmdblock}}
\newenvironment{rmdwarning}
  {\begin{rmdblock}{warning}}
  {\end{rmdblock}}

%%% Use protect on footnotes to avoid problems with footnotes in titles
\let\rmarkdownfootnote\footnote%
\def\footnote{\protect\rmarkdownfootnote}

%%% Change title format to be more compact
\usepackage{titling}

% Create subtitle command for use in maketitle
\newcommand{\subtitle}[1]{
  \posttitle{
    \begin{center}\large#1\end{center}
    }
}

\setlength{\droptitle}{-2em}
  \title{GerminaQuant App}
  \pretitle{\vspace{\droptitle}\centering\huge}
  \posttitle{\par}
  \author{Flavio Lozano Isla \\ Omar Benites Alfaro \\ Marcelo Francisco Pompelli}
  \preauthor{\centering\large\emph}
  \postauthor{\par}
  \predate{\centering\large\emph}
  \postdate{\par}
  \date{17 Julio 2016}

\begin{document}
\maketitle

{
\setcounter{tocdepth}{1}
\tableofcontents
}
\chapter*{Introduction}\label{introduction}
\addcontentsline{toc}{chapter}{Introduction}

\textbf{GerminaQuant} App allows make the calculation for the
germination variables incredibly easy in a interactive applications
build in R \citep{R-base}, based in GerminaR and Shiny \citep{R-shiny}
package. GerminaQuant app is reactive!. Outputs change instantly as
users modify inputs, without requiring a reload the app.

\textbf{Features}

\begin{itemize}
\tightlist
\item
  Allow calculate the princiapal Germination Variables.
\item
  Statistical Analysis for Germination Variables.
\item
  Easy way to plot the results.
\end{itemize}

\chapter{Seed germination process}\label{seed-germination-process}

The physiology and seed technology have provided valuable tools for the
production of high quality seed and treatments and storage conditions
\citep{Marcos-Filho1998}. In basic research, the seeds are studied
exhaustively, and the approach of its biology is performed to fully
exploit the dormancy and germination \citep{Penfield2009}. An important
tool to indicate the performance of a seed lot is the precise
quantification of germination through accurate analysis of the
cumulative germination data \citep{Joosen2010}. Time, speed, homogeneity
and synchrony are aspects that can be measured, and inform the dynamics
of the germination process. These characteristics are interesting not
only for physiologists and seed technologists, but also for ecologist,
since it is possible to predict the degree of success of the species,
based on the seed crop ability to redistribute germination over time,
allowing the recruitment the part of the environment formed seedlings
\citep{Ranal2006}.

\chapter{Germination variables}\label{germination-variables}

\section{Germination (g)}\label{germination-g}

According \citet{GouveaLabouriau1983}, the germinability of a sample of
is the percentage of seeds in which the seed germination process comes
to the end, in experimental conditions by the seminal intrauterine
growth resulting protrusion (or emergence) of a living embryo. In
general, it is presented as percentage, accompanied by some degree of
dispersion, but it is possible to use proportions to one or more samples
may be subjected to statistical tests \citep{CARVALHO2005}.

\[ g=\left(\frac{\sum_{i=1}^kn_1}{N}\right)100 \]

\textbf{Where:} n\textsubscript{i}: number of germinated seed in the
i\textsuperscript{th} time; N: total number of seed in each experimental
unit.

\section{Mean Germination Time (t)}\label{mean-germination-time-t}

It was proposed by Haberlandt in 1875. It is calculated as the weighted
average germination time. The number of germinated seeds at the
intervals established for the collection of data is used as weight. It
is expressed in terms of the same units of time used in the germination
count \citep{Czabator1962}.

\[ t=\frac{\sum_{i=1}^kn_it_i}{\sum_{i=1}^kn_i} \]

\textbf{Where} t\textsubscript{i}: Time of start of experiment tempo to
observation (days or hours); n\textsubscript{i}:number of seed
germinated the i\textsuperscript{th} time (number corresponding to
i\textsuperscript{th} observation); k:the last day of germination.

\section{Mean Germination Rate (v)}\label{mean-germination-rate-v}

The average speed of germination is defined as the reciprocal of the
average time germination \citep{Ranal2006}.

\[ v\ =\frac{1}{i} \]

\textbf{Where} t: mean germination time

\section{Uncertainty Index (U)}\label{uncertainty-index-u}

The uncertainty index (U) is an adaptation of Shannon index measures the
degree of uncertainty in predicting the informational entropy or
uncertainty associated with the distribution of the relative frequency
of germination \citep{GouveaLabouriau1983, Labouriau1983}. Low values of
U indicate frequencies with short peaks, i.e.~the more concentrated the
germination in time. Just a germinated seed changes the value of U. This
means that u measures the degree of germination scattering.

\[ -\sum_{i=1}^kf_i\log_2f_i\ \ \Leftrightarrow\ \ \ f_i=\frac{n_i}{\sum_{i=1}^kn_i}  \]

\textbf{Where} f\textsubscript{i}: relative frequency of germination;
n\textsubscript{i}:number of seed germinated in the time i number
corresponding to i\textsuperscript{th} observation); k:the last day of
germination.

\section{Synchrony Index (Z)}\label{synchrony-index-z}

The Synchrony Index (Z) has been proposed to assess the degree of
overlap between flowering individuals in a population. Adopting the idea
expressed by \citet{primack1985patterns}, Synchrony index is the
synchrony of germination of one seed with other seeds included in the
same replication. When synchrony index = 1, germination of all the seeds
occurs at the same time and when synchrony index = 0, at least two seeds
can germinate one each time. Synchrony index produces a number if and
only if there are two seeds finishing the seed germination process at
the same time. Thus, the value of Z assessments is the grade of overlap
between seed germination.

\[Z=\frac{\sum_{ }^{ }C_{n_1,2}}{N}\ \ \Leftrightarrow\ \ C_{n_1,2}=\frac{n_i\left(n_i-1\right)}{2}\ \ \Leftrightarrow\ \ N=\frac{\sum_{ }^{ }n_i\left(\sum_{ }^{ }n_i-1\right)}{2}\]

\textbf{Where} C\textsubscript{n1},2: combination of germinated seeds in
i\textsuperscript{th} time; n\textsubscript{i}:number of germinated seed
in the time i.

\chapter{Germination Field book}\label{germination-field-book}

For correct analysis and fast data processing is important take account
the data organization and the correct data collection of the germination
process. This section we going to explain how you have to collect and
organize your data.

For data example and layout, you can access and download GerminaQuant
\href{https://docs.google.com/spreadsheets/d/1QziIXGOwb8cl3GaARJq6Ez6aU7vND_UHKJnFcAKx0VI/edit\#gid=667855537}{spreadsheet}.

\section{Data Organization}\label{data-organization}

The field book should have 3 essential parts; the factor columns (red).
It will be use for the experiment treatment and statistical analysis and
It will be according your experimental design; the seed number column
(green) It will indicate the number of seed sown in each experimental
units and the evaluation days column (blue), It will be fill with the
germination values and It will be last according the experiment
programmation and requirements. Figure \ref{fig:dtorg}

You can design your own field book with different names in the column,
remember you need at least one column with factors or treatments, the
column with the number of seed for each experimental unit and the
evaluation days (with prefix) according to the time lapse of your
experiment.

\section{Data Collection}\label{data-collection}

The evaluation of the germination process is obtained of the count of
the germination in each experimental unit and It can be evaluated in
time lapse of hours, days or months in continuous interval of the same
length always beginning with the time zero (ei. Ti00) until the
germination is complete and It can be when for 5 evaluation collection
the values of germination are constant or according the experimental
design.

\chapter{Germination analysis}\label{germination-analysis}

After fish the data collection of the germination process these
information can be processed using GerminaQuant App. The web application
can be used in any device connected to the internet in an interactive
way. The application is compound in tabs (Table \ref{fig:tabs}) that
allow to made the analysis very easy.

\section{GerminaQuant Web
Application}\label{germinaquant-web-application}

\href{https://flavjack.shinyapps.io/germinaquant/}{\includegraphics{germinaquant_files/figure-latex/app-1.pdf}}

\chapter{GerminaQuant data
processing}\label{germinaquant-data-processing}

\section{Field Book}\label{field-book}

For using the GerminaQuant app is necessary that you have a data with
germination values. You can use a data
\href{https://docs.google.com/spreadsheets/d/1QziIXGOwb8cl3GaARJq6Ez6aU7vND_UHKJnFcAKx0VI/edit\#gid=667855537}{sample}
for use GerminaQuant App. Open the link and download the data in csv
format. The document should be in ``csv'' format.

Files \textgreater{}\textgreater{} Download \textgreater{}\textgreater{}
Comma Separated Values (.csv, current sheet)
\textgreater{}\textgreater{} ``GerminaQuant - Sample.csv''

If you have a google account you can clone the document for you and edit
it online and download for your own analysis.

\section{Import Data}\label{import-data}

When you have your field book, you can go
\href{https://flavjack.shinyapps.io/germinaquant/}{GerminaQuant} and go
``Data Import'' tab. Figure \ref{fig:impdt}.

Choose the file in ``csv'' format will be analysed. There are two case
that will have default values if you use the . ``Column with seeds
number'' you have to write the name of the column containing the
information of the number of seed sown in each experimental unit ,
``Prefix of evaluation days'' you have to put the prefix of the name
called for the day for evaluate the germination time lapse.

Below of the parameter for evaluation, you will find the option to
select the parameter for the ``csv'' format file, in such way the file
should have a table form. Figure \ref{fig:csv}

\section{Germination analysis}\label{germination-analysis-1}

If the parameter in the ``Import Data'' tab are correct, in
``Germination analysis'' tab will be appear the values of the eleven
germination variables for each experimental unit. Figure \ref{fig:var}

GerminaQuant app allow to download the file in ``csv'' format with the
calculation of the germination variables. Figure \ref{fig:dwl}

\section{Statistical analysis}\label{statistical-analysis}

In this tab, the app perform a unifactor variance analysis, calculate
the statistical description of the factor, the mean differences through
three mean test: Tukey, Student Newman Keuls and Duncan and made the
graphic for the chosen variable.

Remember, the independiente variables will be the factor in your field
book and the dependent variable will be any of the eleven of germination
variables. Automatically the app will generate the graph for the
variable chosen and give the mean comparison test. The axis label can be
edited manually filling the case in the ``Graphics labels'' section. The
bar graphs represent the mean and central line the standard error.

\section{Germination in Time}\label{germination-in-time}

This Tab allows to visualise the germination process for each factor
included in the field book. Figure \ref{fig:gtime}

The app give two graphics, the first is the germination in percentage in
time lapse and the second the relative germination that calculate the
germination according the total number of seed germinated.

\section{Box Plot}\label{box-plot}

This section allows to plot the distribution of the germination
observation through the box plot (outliers values and overlapping
values) and allows to put label in the graph (Figure \ref{fig:bplot}).
In the section ``Graphics variables'', ``Axis X'' and ``Grouped'' you
should choose between the factor in the the field book, are recommended
in the ``Axis X'' you should choose the factor with more level and
``Grouped'' the factor with less levels, and the ``Axis Y'' any of the
eleven germination variables, in the case you have only one factor
``Axis X'' and ``Grouped'' should have the same factor. The axis label
can be edited manually filling the case in the ``Graphics labels''
section.

\chapter{Demostrative Video}\label{demostrative-video}

\begin{itemize}
\item
  \href{https://drive.google.com/file/d/0B4Ou4jPpNgJbeHVoeG54dGo3bm8/view}{GerminaQuant
  with R console}
\item
  \href{https://drive.google.com/file/d/0B4Ou4jPpNgJbaVJMUFUwc2Z6QnM/view}{GerminaQuant
  web Application}
\end{itemize}

\bibliography{packages,book}


\end{document}
