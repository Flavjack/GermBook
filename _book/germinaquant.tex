\documentclass[]{book}
\usepackage{lmodern}
\usepackage{amssymb,amsmath}
\usepackage{ifxetex,ifluatex}
\usepackage{fixltx2e} % provides \textsubscript
\ifnum 0\ifxetex 1\fi\ifluatex 1\fi=0 % if pdftex
  \usepackage[T1]{fontenc}
  \usepackage[utf8]{inputenc}
\else % if luatex or xelatex
  \ifxetex
    \usepackage{mathspec}
  \else
    \usepackage{fontspec}
  \fi
  \defaultfontfeatures{Ligatures=TeX,Scale=MatchLowercase}
\fi
% use upquote if available, for straight quotes in verbatim environments
\IfFileExists{upquote.sty}{\usepackage{upquote}}{}
% use microtype if available
\IfFileExists{microtype.sty}{%
\usepackage{microtype}
\UseMicrotypeSet[protrusion]{basicmath} % disable protrusion for tt fonts
}{}
\usepackage[margin=1in]{geometry}
\usepackage{hyperref}
\hypersetup{unicode=true,
            pdftitle={GerminaQuant},
            pdfauthor={Flavio Lozano Isla; Omar Benites Alfaro; Marcelo Francisco Pompelli},
            pdfborder={0 0 0},
            breaklinks=true}
\urlstyle{same}  % don't use monospace font for urls
\usepackage{natbib}
\bibliographystyle{apalike}
\usepackage{longtable,booktabs}
\usepackage{graphicx,grffile}
\makeatletter
\def\maxwidth{\ifdim\Gin@nat@width>\linewidth\linewidth\else\Gin@nat@width\fi}
\def\maxheight{\ifdim\Gin@nat@height>\textheight\textheight\else\Gin@nat@height\fi}
\makeatother
% Scale images if necessary, so that they will not overflow the page
% margins by default, and it is still possible to overwrite the defaults
% using explicit options in \includegraphics[width, height, ...]{}
\setkeys{Gin}{width=\maxwidth,height=\maxheight,keepaspectratio}
\IfFileExists{parskip.sty}{%
\usepackage{parskip}
}{% else
\setlength{\parindent}{0pt}
\setlength{\parskip}{6pt plus 2pt minus 1pt}
}
\setlength{\emergencystretch}{3em}  % prevent overfull lines
\providecommand{\tightlist}{%
  \setlength{\itemsep}{0pt}\setlength{\parskip}{0pt}}
\setcounter{secnumdepth}{5}
% Redefines (sub)paragraphs to behave more like sections
\ifx\paragraph\undefined\else
\let\oldparagraph\paragraph
\renewcommand{\paragraph}[1]{\oldparagraph{#1}\mbox{}}
\fi
\ifx\subparagraph\undefined\else
\let\oldsubparagraph\subparagraph
\renewcommand{\subparagraph}[1]{\oldsubparagraph{#1}\mbox{}}
\fi

%%% Use protect on footnotes to avoid problems with footnotes in titles
\let\rmarkdownfootnote\footnote%
\def\footnote{\protect\rmarkdownfootnote}

%%% Change title format to be more compact
\usepackage{titling}

% Create subtitle command for use in maketitle
\newcommand{\subtitle}[1]{
  \posttitle{
    \begin{center}\large#1\end{center}
    }
}

\setlength{\droptitle}{-2em}
  \title{GerminaQuant}
  \pretitle{\vspace{\droptitle}\centering\huge}
  \posttitle{\par}
  \author{Flavio Lozano Isla \\ Omar Benites Alfaro \\ Marcelo Francisco Pompelli}
  \preauthor{\centering\large\emph}
  \postauthor{\par}
  \predate{\centering\large\emph}
  \postdate{\par}
  \date{09 October 2016}

\usepackage{booktabs}
\usepackage{longtable}
\usepackage{framed,color}
\definecolor{shadecolor}{RGB}{248,248,248}

\ifxetex
  \usepackage{letltxmacro}
  \setlength{\XeTeXLinkMargin}{1pt}
  \LetLtxMacro\SavedIncludeGraphics\includegraphics
  \def\includegraphics#1#{% #1 catches optional stuff (star/opt. arg.)
    \IncludeGraphicsAux{#1}%
  }%
  \newcommand*{\IncludeGraphicsAux}[2]{%
    \XeTeXLinkBox{%
      \SavedIncludeGraphics#1{#2}%
    }%
  }%
\fi

\newenvironment{rmdblock}[1]
  {\begin{shaded*}
  \begin{itemize}
  \renewcommand{\labelitemi}{
    \raisebox{-.7\height}[0pt][0pt]{
      {\setkeys{Gin}{width=3em,keepaspectratio}\includegraphics{images/#1}}
    }
  }
  \item
  }
  {
  \end{itemize}
  \end{shaded*}
  }
\newenvironment{rmdnote}
  {\begin{rmdblock}{note}}
  {\end{rmdblock}}
\newenvironment{rmdcaution}
  {\begin{rmdblock}{caution}}
  {\end{rmdblock}}
\newenvironment{rmdimportant}
  {\begin{rmdblock}{important}}
  {\end{rmdblock}}
\newenvironment{rmdtip}
  {\begin{rmdblock}{tip}}
  {\end{rmdblock}}
\newenvironment{rmdwarning}
  {\begin{rmdblock}{warning}}
  {\end{rmdblock}}

\begin{document}
\maketitle

{
\setcounter{tocdepth}{1}
\tableofcontents
}
\chapter*{Introduction}\label{introduction}
\addcontentsline{toc}{chapter}{Introduction}

\textbf{GerminaQuant} allows make the calculation of the germination
variables incredibly easy in an interactive applications build in R
\citep{R-base}, based in GerminaR and Shiny \citep{R-shiny} package.
GerminaQuant app is reactive!. Outputs change instantly as users modify
inputs, without requiring a reload the app. The principal features of
the application allow calculate the princiapal germination Variables,
statistical analysis and easy way to plot the results.

\chapter{Evaluation of the seed germination
process}\label{evaluation-of-the-seed-germination-process}

The physiology and seed technology have provided valuable tools for the
production of high quality seed and treatments and storage conditions
\citep{Marcos-Filho1998}. In basic research, the seeds are studied
exhaustively, and the approach of its biology is performed to fully
exploit the dormancy and germination \citep{Penfield2009}. An important
tool for indicate the performance of a seed lot is the precise
quantification of germination through accurate analysis of the
cumulative germination data \citep{Joosen2010}. Time, velocity,
homogeneity, uncertainty and synchrony are measurements that inform the
dynamics of the germination process. It is interesting not only for
physiologists and seed technologists, but also for environmentalists,
since it is possible to predict the degree of success of the species,
based on the seed crop ability to redistribute germination over time,
allowing the recruitment of part of the seedlings formed
\citep{Ranal2006}.

\chapter{Germination variables}\label{germination-variables}

\section{Germinability (G)}\label{germinability-g}

According to \citet{Labouriau1983a}, germinability is the percentage of
seeds in which the germination process comes to the end in the
experimental conditions by the intraseminal growth, resulting in the
protrusion (or emergence) of a living embryo.

\[ g=\left(\frac{\sum_{i=1}^kn_1}{N}\right)100 \ \ (in\ percentage;\ expressed\ in\ the\ program\ as\ GRP)\]

\textbf{Where:} \(n_i\): number of germinated seed in the \(i^{th}\)
time; N: total number of seed in each experimental unit.

\section{\texorpdfstring{Mean Germination Time (t) and Standard
Deviation
(S\textsubscript{t})}{Mean Germination Time (t) and Standard Deviation (St)}}\label{mean-germination-time-t-and-standard-deviation-st}

The mean germination time was proposed by Haberlandt in 1875
\citep{Labouriau1983a} and used by \citet{Czabator1962} as mean length
of incubation time (see \citet{Ranal2006} for other details). It is
calculated as the weighted mean of the germination time (hour, day or
other time unit). The number of germinated seeds at the intervals
established for the data collection is used as weight \citep{Ranal2006}.

\[ \overline{t}=\frac{\sum_{i=1}^kn_it_i}{\sum_{i=1}^kn_i} \ (in\ hour,\ day\ or\ other\ time\ unit; \ expressed\ in\ the\ program\ as\ MGT)\]

\[S_t=\sqrt{\left\{\frac{\left\{\sum_{_{i=1}}^kn_i\left(t_i-\overline{t}\right)^2\right\}}{\sum_{_{i=1}}^kn_i-1}\right\}} \ \ (in\ the\ same\ time\ unit\ of\ the\ mean\ germination\ time;\ expressed\ in\ the\ program\ as\ SDG)\]

\textbf{Where} \(\bar{t}_i\): time from the start of the experiment to
the \(i^{th}\) observation (hour, day or other time unit); \(n_i\):
number of germinated seeds in the \(i^{th}\) time (not the accumulated
number) and \(k\): the last time of observation.

\section{Coefficient of Variation of the Germination
Time}\label{coefficient-of-variation-of-the-germination-time}

\[CV_t=\frac{\left\{S_t\right\}}{\overline{t}}100\ \left(in\ the\ same\ tine\ unit\ of\ the\ mean\ ger\min ation\ time;\ \exp ressed\ in\ the\ program\ as\ SDG\right)\]
\textbf{where} \(S_t\): standard deviation of the germination time and
\(\overline{t}\): mean germination time.

\section{Mean Germination Rate (v)}\label{mean-germination-rate-v}

The mean germination rate is defined as the reciprocal of the mean
germination time, since the mean germination rate increases and
decreases with \(\frac{1}{\overline{t}}\), not with \(\overline{t}\)
\citep{Labouriau1983b}.

\[ \overline{v}=\frac{1}{\overline{t}}\ \left(in\ hours^{-1},\ day^{-1}\ or\ other\ reciprocal\ time\ unit,\ \exp ressed\ in\ the\ program\ as\ MGR\right) \]

\section{Uncertainty Index (U)}\label{uncertainty-index-u}

This measurement is an adaptation of the Shannon index and measures the
degree of uncertainty associated to the distribution of the relative
frequency of germination \citep{labouriau1976physiology}. Low values
indicate more synchronized germination \citep{Ranal2006}.

\[U\ =\ -\sum_{i=1}^kf_i\log_2f_i\ \ \left(in\ bit;\ \exp ressed\ in\ the\ program\ as\ UNC\right),\ being\ f_i=\frac{n_i}{\sum_{i=1}^kn_i}\]

\textbf{Where} \(f_i\): relative frequency of germination; \(n_i\):
number of seed germinated in the \(i^{th}\) time, and \(k\): the last
day of germination.

\section{Synchronyzation Index (Z)}\label{synchronyzation-index-z}

This index was proposed by \citet{primack1985patterns} to assess the
degree of overlapping of flowering among individuals in a population and
\citet{Ranal2006} adopted it for seed germination. The synchrony of
germination of one seed with another assumed \(Z=1\) when the
germination of all seeds occur at the same time and \(Z = 0\) when at
least two seeds can germinate, one at each time.

\[Z=\frac{\sum_{ }^{ }C_{n_1,2}}{N}\ \left(a\dim ensional\ measurement;\ \exp ressed\ in\ the\ program\ as\ SYN\right);\ being\ C_{n_1,2}=\frac{n_i\left(n_i-1\right)}{2}\ ;\ N=\frac{\sum_{ }^{ }n_i\left(\sum_{ }^{ }n_i-1\right)}{2}\]

\textbf{Where} \(C_{n_i,2}\): combination of germinated seeds in
\(i^{th}\) time, two together, and \(n_i\): number of germinated seed in
the \(i^{th}\).

\section{Limits and units of the germination variables in
GerminaQuant}\label{limits-and-units-of-the-germination-variables-in-germinaquant}

\begin{table}

\caption{\label{tab:varsum}Germination variables evaluated in GerminaQuant and limits according @Ranal2006 ; where: \(n_i\), number of seed germinated in \(i^{nth}\) time ; \(K\), the last day of the avaliation process for germination.}
\centering
\begin{tabular}[t]{llll}
\toprule
variables & abbreviation & limits & units\\
\midrule
germinated seed number & GSN & \textbackslash{}(0 \textbackslash{}le n \textbackslash{}le n\_i\textbackslash{}) & \textbackslash{}(count\textbackslash{})\\
germinability & GNP & \textbackslash{}(0 \textbackslash{}le g \textbackslash{}le 100\textbackslash{}) & \textbackslash{}(\textbackslash{}\%\textbackslash{})\\
germination asin & ASG & \textbackslash{}(0 \textbackslash{}le arsin \textbackslash{}le 1\textbackslash{}) & \textbackslash{}(grade\textbackslash{})\\
mean germination time & MGT & \textbackslash{}(0 \textbackslash{}le t \textbackslash{}le k\textbackslash{}) & \textbackslash{}(time\textbackslash{})\\
germination speed & SPG & \textbackslash{}(0 < g\_s \textbackslash{}le 100\textbackslash{}) & \textbackslash{}(\textbackslash{}\%\textbackslash{})\\
\addlinespace
mean germination rate & MGR & \textbackslash{}(0 < v \textbackslash{}le 1\textbackslash{}) & \textbackslash{}(time\textasciicircum{}\{-1\}\textbackslash{})\\
Synchronyzation Index & SYN & \textbackslash{}(0 \textbackslash{}le Z \textbackslash{}le 1\textbackslash{}) & \textbackslash{}(-\textbackslash{})\\
Uncertainty Index & UNC & \textbackslash{}(0 \textbackslash{}le U \textbackslash{}le log\_2 n\_i\textbackslash{}) & \textbackslash{}(bit\textbackslash{})\\
germination standard deviation & SDG & \textbackslash{}(0 < s\_t\textasciicircum{}2  \textbackslash{}le \textbackslash{}infty\textbackslash{}) & \textbackslash{}(time\textasciicircum{}2\textbackslash{})\\
germination variance & VGT & \textbackslash{}(0 < s\_t  \textbackslash{}le \textbackslash{}infty\textbackslash{}) & \textbackslash{}(time\textbackslash{})\\
Coefficient of variation & CVG & \textbackslash{}(0 < CV\_t  \textbackslash{}le \textbackslash{}infty\textbackslash{}) & \textbackslash{}(\textbackslash{}\%\textbackslash{})\\
\bottomrule
\end{tabular}
\end{table}

\chapter{Germination field book}\label{germination-field-book}

For correct analysis and fast data processing is important to take into
account that the data organization and the correct data collection of
the germination process is essential. In this section, we going to
explain how can you collect and organize your data.

For data example and layout, you can access and download GerminaQuant
\href{https://docs.google.com/spreadsheets/d/1QziIXGOwb8cl3GaARJq6Ez6aU7vND_UHKJnFcAKx0VI/edit\#gid=667855537}{spreadsheet}.

\section{Data Organization}\label{data-organization}

The field book should have three essential parts. The treatment column
(red), according to the experimental design (1); the seed number column
(green) (2), and the observation moment column (blue) (3). In the green
column you will indicate the number of seeds sown in each experimental
unit and in the blue columns the germination values (Figure
\ref{fig:dtorg}). You can design your own field book with different
names in the column.

\section{Data Collection}\label{data-collection}

The evaluation of the germination process is obtained of the count of
the germination in each experimental unit and It can be evaluated in
time lapse of hours, days or months in continuous interval of the same
length always beginning with the time zero (ei. Ti00), until the end of
the germination process or according to the researcher criteria

\chapter{Germination analysis}\label{germination-analysis}

After the data collection, the information can be processed using
GerminaQuant App. The web application can be used in any device,
connected to the internet, in an interactive way. The application is
compound in tabs (Table \ref{tab:tabs}) that allow to make the analysis
very easy.

\begin{table}

\caption{\label{tab:tabs}Name and description of each tab of GerminaQuant to evaluate and analyze the germination process}
\centering
\begin{tabular}[t]{ll}
\toprule
Tabs & Description\\
\midrule
Introduction & Presentation of GerminaQuant and description of the germination variables\\
Data Import & Allow to upload the fieldbook, visualize all the data and chose the parameter for the analysis .\\
Germination Analysis & Calculate automatically the germination variables and export the data file.\\
Statisdtical Analysis & Allow to chose the variables according the experimental design for analysis of variance and summarise the information\\
Multi Plot & Plot the mean comparison test for the variables selected in diffrente plots: bar, line and boxplot.\\
Germination in Time & Selecting the treatment, allows plotting the germination process.\\
\bottomrule
\end{tabular}
\end{table}

\chapter{GerminaQuant data
processing}\label{germinaquant-data-processing}

\section{Field Book}\label{field-book}

For using the GerminaQuant is necessary that you have a data with
germination values. You can use a following data
\href{https://docs.google.com/spreadsheets/d/1QziIXGOwb8cl3GaARJq6Ez6aU7vND_UHKJnFcAKx0VI/edit\#gid=667855537}{sample}.
Open the link and download the data in csv format.

Files \textgreater{}\textgreater{} Download \textgreater{}\textgreater{}
Comma Separated Values (.csv, current sheet)
\textgreater{}\textgreater{} ``GerminaQuant - Sample.csv''

If you have a google account you can clone the document for you and edit
it online and download for your own analysis.

\section{Import Data}\label{import-data}

When you have your field book, you can go
\href{https://flavjack.shinyapps.io/germinaquant/}{GerminaQuant} and go
``Data Import'' tab. Figure \ref{fig:impdt}.

Choose the file in ``csv'' format it will be analysed. In ``Column with
seeds number'' you have to write the name of the column containing the
information of the number of seed sown in each experimental unit,
``Prefix of evaluation days'' you have to put the prefix of the name
called for the day for evaluate the germination time lapse.

Below of the parameter for evaluation, you will find the option to
select the parameter for the ``csv'' format file. In such way the file
should have a table form. Figure \ref{fig:csv}

\section{Indices calculation}\label{indices-calculation}

If the parameter in the ``Import Data'' tab are correct, in
``Germination analysis'' tab will be performed and the values of the
germination variables for each experimental unit will be show. Table
\ref{tab:varsum}. GerminaQuant app allows downloading the file in
``csv'' format with the calculation of the germination variables. Figure
\ref{fig:dwl}

\section{Statistical analysis}\label{statistical-analysis}

In this tab, the app perform a factorial variance analysis, calculate
the statistical description of the factor, the mean differences through
Student Newman Keuls \citep{R-agricolae} and made the graphics
\citep{R-ggplot2} for the chosen variable.

Remember, the independent variables will be the factor in your field
book and the dependent variable will be any of the germination
variables. Automatically the app will generate the graphs for the
variable chosen and give the mean comparison test. The axis label can be
edited manually filling the case in the ``Graphics labels'' section. The
bar and line graphs are represented by the mean and central line is the
standard error.

\section{Germination in Time}\label{germination-in-time}

This Tab allows to visualize the germination process included in the
field book. Figure \ref{fig:gtime}

The app give two graphics, the first is the germination in percentage in
time lapse and the second the relative germination that calculate the
germination according the total number of seed germinated.

\bibliography{book,packages}


\end{document}
